\chapter{Zaključak}
\label{pog:zakljucak}

S obzirom na zahtjeve uređaja, sklopovlje prototipa uređaja razvijeno u ovome radu bi se moglo smatrati skoro gotovim. Jedini dio sustava s kojim je došlo do poteškoća, gledajući sa strane dizajna, jest napajanje, a i kasnije su spomenute poteškoće otklonjene ispravnim dizajnom. Drugi problem je bio dvostrana montaža pločice. Međutim, imajući na umu da se u ovom radu pločica montirala ručno, moguće je zaključiti da do ovakvih problema neće doći u uvjetima profesionalne proizvodnje. Sljedeći korak bi bio ovaj prototip unaprijediti tako da bude što manje invazivan na udobnost korisnika kako bi se u potpunosti ispunili zahtjevi na uređaj.

Za početak, potrebno je dizajnirati elektrode za mjerenje impedancije kože. U sadašnjem stadiju, impedancija se mjeri preko samoljepljivih elektroda spojenih na uređaj s pomoću veoma duge žice. Očito je da je takva metoda u potpunosti neprihvatljiva za korisnika koji bi ovaj uređaj trebao nositi cijeli dan uokolo dok obavlja razne zadatke. Cilj je napraviti narukvicu koja će elektrode imati izložene na kućištu narukvice, koja će tako ostvarivati kontakt s kožom. Elektrode bi se onda montirale na donju stranu pločice, kao što je zamišljeno s PPG senzorom.

Nadalje, potrebno je smanjiti fizičke dimenzije pločica. To će se ostvariti uklanjanjem testnih točaka i kratkospojnika koji zauzimaju veliku količinu prostora, kako radi svoje veličine, tako radi svojih velikih međusobnih razmaka na pločici koji su bili potrebni za ispravno i jednostavno testiranje. Još jedan značajniji način na koji će se smanjiti veličina pločice jest korištenje manjih komponenata. To su većinom pasivne komponente, koje su trenutačno u 0603 kućištu, a odokativnom procjenom, korištenje kućišta 0402 bi već smanjile dimenzije pločica za više od pola. Tu je također priključak za SD karticu, koji zauzima površinu veličine sustava za bežičnu komunikaciju. Korištenjem microSD kartice ta površina će se smanjiti četiri puta. Također bi bilo moguće smanjiti sustav za bežičnu komunikaciju tako da se dizajnira vlastiti sustav od diskretnih komponenata, za razliku od modula koji se trenutačno nalazi na pločici.

Također je potrebno promijeniti mikrofon. Naime, mikrofon se sada nalazi montiran direktno na pločicu središnjeg sustava. To je nepraktično jer bi korisnik morao nositi kutiju značajne veličine montiranu negdje blizu glave radi boljeg primitka zvuka. To bi se moglo riješiti klasičnim mikrofonom u bubici, koja je dizajnirana upravo tako da se zakači što bliže glavi, a da pritom što manje smeta. Taj bi se mikrofon onda mogao spajati na središnji sustav žicom koji se nalazi negdje blizu struka korisnika ili gdje god ga korisnik želi montirati.

Iako je filtriranje signala sa sustava za mjerenje impedancije kože izvedeno softverski, bilo bi bolje to učiniti sklopovski. Koristi se manje procesorskih resursa i time se više vremena prepušta mjerenju. To se može napraviti klasičnim niskopropusnim filtrom prvoga reda. Impedancija kože se vrlo sporo mijenja, pa će takav filtar biti dovoljno dobar za ovu situaciju.

Naravno, nema smisla da korisnik uokolo hoda s golim pločicama, iz očitih razloga. Potrebno je dizajnirati kućište i način na koji će korisnik uređaj staviti na sebe, a da mu on što manje smeta dok ga nosi.