\chapter{Glavna ploča}
\label{pog:mainboard}

Sustav se sastoji od dva uređaja koji rade u simbiozi. Glavna ploča služi za snimanje, obradu i pohranu glasovnih podataka i obradu i pohranu biomedicinskih parametara. Za snimanje biomedicinskih parametara koristi se narukvica. U ovom poglavlju se opisuje glavna ploča.

Zahtjevi na glavnu ploču su sljedeći:
\begin{itemize}
    \item mikrokontroler \engl{Microcontroler Unit, MCU}, dovoljno moćan za pokretanje neuralnih mreža i obradu podataka,
    \item konektor za SD karticu,
    \item bežična komunikacija putem Wi-Fi ili Bluetooth sučelja,
    \item praćenje vremena putem RTC-a,
    \item mikrofon za prikupljanje govora korisnika,
    \item sučelja za testiranje i prženje koda na mikrokontroler,
    \item napajanje i punjenje baterije preko USB C priključka,
    \item baterijsko napajanje putem litij-ionske baterije.
\end{itemize}
Blok dijagram glavne ploče je prikazan na slici \ref{slk:MB_BD}
\begin{figure}[htb]
    \centering
    \includegraphics[width=\textwidth]{Figures/MB_BD.png}
    \caption{Blok dijagram glavne ploče}
    \label{slk:MB_BD}
\end{figure}
U daljnjem tekstu ovog poglavlja opisane su odabrane komponente, kao i razlog njihova odabira, način, razlozi i proračuni dizajna pojedinih podsustava, te dizajn, proizvodnja i testiranje PCB-a.

\section{Mikrokontroler}
Za mikrokontroler odabran je STM32F746VG baziran na Cortex-M7 arhitekturi koji integrira funkcionalnosti digitalne obrade signala, bogat sa svim potrebnim periferijama za integriranje s ostatkom sustava i dovoljno procesorske snage za obavljanje zadanog zadatka. Također, programska potpora je razvijena na razvojnom sustavu BLABLABLA, pa je ovaj mikrokontroler odabran radi lakšeg razvoja cjelokupnog sustava. Shema napajanja mikrokontrolera prikazana je na slici \ref{slk:MCU_PS}, a shema spajanja mikrokontrolera s ostatkom sustava prikazana je na slici \ref{slk:MCU_PE}.

\begin{figure}[hbt]
    \centering
    \includegraphics[width=\textwidth]{Figures/MCU_02.png}
    \caption{Shema napajanja mikrokontrolera}
    \label{slk:MCU_PS}
\end{figure}

Shema napajanja napravljena je prema uputama proizvođača \cite{stmicroelectronics:an4661}. S obzirom na to da na ovoj ploči nema analognih signala, nije potrebno raditi analogno-digitalnu pretvorbu, pa su stezaljke za napajanje analognog dijela mikrokontrolera spojene sa stezaljkama za napajanje digitalnog dijela. Također, nije potrebna precizna naponska referenca, a baterijskim napajanjem će upravljati vanjski čip, pa su te dvije stezaljke spojene na napajanje od +3.3 V.
\newpage

\begin{figure}[hbt]
    \centering
    \includegraphics[width=\textwidth]{Figures/MCU_01.png}
    \caption{Shema periferije mikrokontrolera}
    \label{slk:MCU_PE}
\end{figure}

Postavljene su dvije svjetleće diode za pomoć pri programiranju i tipka za reset mikrokontrolera. Kod određivanja korištenih stezaljki za pomoć je korišteno razvojno okruženje STM32CubeIDE. Dizajn oscilatora opisan je u priručniku proizvođača STMicroelectronics \cite{stmicroelectronics:an2867}. Oscilator koji mikrokontroler koristi je Pierceov oscilator (slika \ref{slk:PIERCE}).
\begin{figure}[hbt]
    \centering
    \includegraphics[width=10cm]{Figures/pierce.PNG}
    \caption{Shema Piercovog oscilatora \cite{stmicroelectronics:an2867}}
    \label{slk:PIERCE}
\end{figure}

\subsection{Pierceov oscilator}
Pierceov oscilator se sastoji od invertera \textit{Inv}, koji radi kao pojačalo, kristala \textit{Q}, otpornika u povratnoj vezi \textit{R\textsubscript{F}}, vanjskog otpornika za ograničenje izlazne struje invertera \textit{R\textsubscript{Ext}}, vanjskih kapaciteta opterećenja \textit{C\textsubscript{L1}} i \textit{C\textsubscript{L2}}, parazitskog kapaciteta PCB-a i kapaciteta između stezaljki mikrokontrolera \textit{C\textsubscript{s}}.

Uloga otpornika \textit{R\textsubscript{F}} je da natjera inverter da se ponaša poput pojačala. Otpornik povratne veze je spojen između izlaza i ulaza pojačala čime se ulaz i izlaz drže na istom naponi i prisiljava pojačalo da radi u linearnom području rada. Ovaj otpornik je integriran u mikrokontroleru zajedno s pojačalom.

Kapacitet opterećenja je ukupni kapacitet povratne veze oscilatora i mora biti jednak je kapacitetu između stezaljki kristala kako bi oscilator prooscilirao. Kapacitet opterećenja specificira proizvođač kristala i označava se s \textit{C\textsubscript{L}}. Vanjskim kondenzatorima \textit{C\textsubscript{L1}} i \textit{C\textsubscript{L2}} se postavlja kapacitet opterećenja povratne veze kako bi odgovarao kapacitetu opterećenja kristala. Kapacitet opterećenja računa se prema sljedećoj jednadžbi:
\begin{equation} \label{eq:CLOAD}
    C_L=\frac{C_{L1}\cdot C_{L2}}{C_{L1}+C_{L2}}+C_s
\end{equation}

Kod dizajna oscilatora potrebno je izračunati kritično pojačanje petlje:
\begin{equation} \label{eq:GMCRIT}
    g_{mcrit}=4\cdot ESR \cdot {(2\pi F)}^2\cdot {(C_0 + C_L)}^2
\end{equation}
gdje je \textit{F} frekvencija oscilatora, \textit{ESR} serijski otpor kristala i \textit{C\textsubscript{0}} serijski kapacitet kristala. Ovaj parametar je potrebno izračunati kako bi se moglo provjeriti hoće li se oscilator upaliti i prooscilirati. Dobiveni podatak se uspoređuje sa specificiranim vrijednostima transkonduktivnosti \textit{g\textsubscript{m}} u dokumentaciji mikrokontrolera. Da bi oscilator proradio mora se proračunati margina pojačanja i treba vrijediti:
\begin{equation} \label{eq:GMARGIN}
    {gain}_{margin}=\frac{g_m}{g_{mcrit}}>5
\end{equation}

Kako ne bi došlo do kvara kristala potrebno je ograničiti snagu koja se na njemu disipira s pomoću vanjskog otpornika \textit{R\textsubscript{Ext}}. Maksimalna snaga koja se može disipirati na kristalu naznačena je u dokumentaciji proizvođača. Ovaj otpornik s kondenzatorom \textit{C\textsubscript{L2}} formira niskopropusni filtar kako bi oscilator proradio na osnovnoj frekvenciji, a ne na višim harmonicima. Ako snaga disipirana na kristalu bude veća od maksimalne dozvoljene onda je vanjski otpornik obavezan i mora se proračunati, u suprotnom ga nije potrebno stavljati. Vrijednost otpornika se računa na sljedeći način:
\begin{equation} \label{eq:REXT}
    R_{Ext}=\frac{1}{2\pi F C_{L2}}
\end{equation}

\subsection{Dizajn oscilatora}

Oscilator je vidljiv na slici \ref{slk:MCU_PE}. Odabran je kristal NX8045GB proizvođača NDK. Njegove karakteristike su sljedeće:
\begin{itemize}
    \item \textit{C\textsubscript{L}} = 2 pF
    \item \textit{ESR} = 200 $\Omega$
    \item \textit{F} = 8 MHz
\end{itemize}
\textit{C\textsubscript{0}} nije naznačen pa se uzima vrijednost 0. Uzimajući za parazitni kapacitet \textit{C\textsubscript{s}} = 2 pF, i koristeći formulu \ref{eq:CLOAD} dobiju se vrijednosti kondenzatora \textit{C\textsubscript{L1}} = \textit{C\textsubscript{L2}} = 12 pF. Odabir parazitnog kapaciteta je odokativan jer se ne može znati unaprijed bez mjerenja dovršene tiskane pločice. Kod odabira kondenzatora potrebno je obratiti pažnju na dielektrik kondenzatora i tolerancije. Kako bi frekvencija oscilatora bila što stabilnija potrebno je koristiti temperaturno stabilan dielektrik, odnosno kondenzatore klase 1. Korišteni kondenzatori imaju C0G dielektrik. Koristeći jednadžbu \ref{eq:GMCRIT} dobiva se ${g_{mcrit} = 0.1294 \quad \textrm{mA/V}}$. Iz dokumentacije mikrokontrolera se dobiva ${g_m = 1\quad \textrm{mA/V}}$. Iz uvjeta \ref{eq:GMARGIN} dobiva se ${{gain}_{margin}=\nolinebreak 7.73}$, čime je uvjet zadovoljen. S obzirom na to da nije moguće odrediti koliko će se kristal grijati, za vanjski otpornik postavljen je otpornik vrijednosti 0 $\Omega$, pa u slučaju prevelike disipacije snage na kristalu moguće je na njegovo mjesto zalemiti otpornik odgovarajuće vrijednosti prema jednadžbi \ref{eq:REXT}.

\section{RTC}

S obzirom na to da uređaj treba uskladiti podatke s mikrofona i narukvice potrebno je precizno praćenje vremena. U tu svrhu dodan je vanjski RTC PCF8653 proizvođača NXP \cite{nxp:pcf8654}. Za ovaj čip postoji već razvijena programska podrška u ZephyrOS operacijskom sustavu za rad u stvarnom vremenu, pa je razvoj programske potpore za uređaj znatno olakšana. Shema RTC-a prikazana je na slici \ref{slk:RTC}.
\begin{figure}[hbt]
    \centering
    \includegraphics[width=\textwidth]{Figures/RTC.png}
    \caption{Shema RTC-a}
    \label{slk:RTC}
\end{figure}
S obzirom na to da se litij-ionska baterija, koja napaja cijeli uređaj, može isprazniti, otkopčati ili na neki drugi način se može prekinuti napajanje s vanjske baterije, RTC se napaja iz litijske baterije kako praćenje vremena ne bi bilo izgubljeno. S obzirom na to da je napon litijske baterije 3 V, a napon ostatka sustava 3.3 V, na I\textsuperscript{2}C linije dodane su Zener diode s probojnim naponom od 2.7 V, čime se maksimalni napon ograničava kako ne bi došlo do oštećenja čipa tijekom komunikacije s mikrokontrolerom.

\section{Bežična komunikacija}
\begin{figure}[hbt]
    \centering
    \includegraphics[width=\textwidth]{Figures/WIRELESS.png}
    \caption{Shema podsustava za bežičnu komunikaciju}
    \label{slk:WIFI}
\end{figure}
Shema podsustava za bežičnu komunikaciju prikazana je na slici \ref{slk:WIFI}. Radi lakšeg razvoja odabran je razvojni sustav ESP32-C3-WROOM-02 proizvođača Espressif Systems. Shema je razvijena prema preporukama proizvođača \cite{espressif:wroom02}. Dodan je još jedan kratkospojnik za ispitivanje funkcionalnosti reseta sustava.

\section{SD kartica i konektori}
\begin{figure}[hbt]
    \centering
    \includegraphics[width=\textwidth]{Figures/SD.png}
    \caption{Shema konektora SD kartice}
    \label{slk:SD}
\end{figure}

Za pohranu podataka na uređaju se nalazi SD kartica standardne veličine. Razlog odabira ove veličine jest taj da uređaj onda podržava i standardnu SD karticu i adapter za SD kartice manje veličine. Shema SD konektora prikazana je na slici \ref{slk:SD}. Konektor ima ESD zaštitu u obliku diode D401 i filtriranje napajanja preko mreže koja se sastoji od kondenzatora i feritne perle. Na svim komunikacijskim linijama se nalaze pritezni otpornici, a linija za takt ima terminacijski otpor od 49.9 $\Omega$ kako bi se suzbilo odzvanjanje. Kako CD stezaljka za detekciju spojene kartice ne bi ostala plutajuća dodan je pritezni otpornik od 1 M$\Omega$. Razlog odabira tako velikog otpora je unutarnji pritezni otpornik prema napajanju na SD karticama, pa se velikim otporom suzbija efekt naponskog djelila.

Za prženje korisničkog programa na mikrokontroler potreban je konektor za SWO sučelje. Dodatno, za testiranje programske podrške i komunikaciju s računalom potreban je konektor za UART sučelje. Sheme konektora prikazane su na slikama \ref{slk:UART_SWO}.
\begin{figure}[hbt]
    \centering
    \includegraphics[width=6 cm]{Figures/SWO_UART.png}
    \caption{Shema konektora za UART i SWO sučelje}
    \label{slk:UART_SWO}
\end{figure}

Za mikrofon se koristi evaluacijska pločica STEVAL-MIC003V1 (slika \ref{slk:STEVAL}) na kojoj se nalazi MEMS mikrofon IMP34DT05 proizvođača STMicroelectronics \cite{stmicroelectronics:steval}. Za ovaj mikrofon također postoji razvijena programska podrška unutar ZephyrOS operacijskog sustava za rad u stvarnom vremenu. Ove evaluacijske pločice na sebi imaju montirane muške konektore s razmakom od 7 mm, pa se ovdje koriste ženski konektori (slika \ref{slk:MEMS}).
\begin{figure}[hbt]
    \centering
    \includegraphics[width=6 cm]{Figures/STEVAL.jpg}
    \caption{STEVAL-MIC003V1}
    \label{slk:STEVAL}
\end{figure}
\begin{figure}[hbt]
    \centering
    \includegraphics[width=6 cm]{Figures/MEMS.png}
    \caption{Konektori za MEMS}
    \label{slk:MEMS}
\end{figure}
Kraj konektora su postavljene testne točke za promatranje signala preko logičkog analizatora u slučaju da postoje poteškoće tijekom programiranja ili rada.

\section{Potrošnja}

Sada kada su sve komponente odabrane, moguće je napraviti proračun potrošnje kako bi se mogao dizajnirati sustav napajanja. Napravljena je tablica potrošnje za sustave koji se napajaju sa 3.3 V (tablica \ref{tab:MB3V3}). Za maksimalne i minimalne vrijednosti potrošnje uzeti su podaci iz dokumentacije komponenata, a prosječna potrošnja je uzeta odokativno jer je nemoguće znati prosječnu potrošnju bez mjerenja.
\begin{table}[htbp]
    \centering
    \caption{Potrošnja struje za sustave koji se napajaju sa 3.3 V}
    \begin{tabular}{lccc} \hline
           & Min. [mA] & Avg. [mA] & Max. [mA] \\
      MCU  & 0,00 & 60,00 & 320,00 \\
      RTC  & 0,00 & 0,80 & 50,00 \\
      SD Card & 1,25 & 25,00 & 100,00 \\
      MEMS & 0,00 & 0,65 & 10,00 \\
      Wireless & 13,00 & 82,00 & 350,00 \\ \hline
      Total on SYS & 14,25 & 168,45 & 830,00 \\ \hline
    \end{tabular}%
    \label{tab:MB3V3}%
\end{table}%
Sada je moguće odabrati prikladne komponente za napajanje.

\section{Napajanje i punjač baterije}
Za punjač baterije odabran je BQ24166 proizvođača Texas Instruments (slika \ref{slk:BQ24166}). Ovaj integrirani krug u sebi ima integriran sustav za upravljanje tokom snage \cite{ti:bq24166}. BQ24166 se može napajati s dva ulaza; ulaz za USB ili ulaz za druge vrste napajanja (AC/DC adapter, DC laboratorijski izvor napajanja, itd.), a da pritom u isto vrijeme puni bateriju i na svom izlazu daje napon baterije, s tim da izlazni napon neće pasti ispod 3.5 V. U tu svrhu u čip je ugrađen silazni prekidački regulator napona, kako bi se kod punjenja baterije konstantnim naponom dobio izlazni napon od 4.2 V potreban za punjenje. Ako na ulaz čipa nije spojeno ništa, onda se na izlaz direktno prosljeđuje napon baterije. U slučaju da napon baterije padne ispod 3.5 V, a da pritom ništa nije spojeno na ulaz čipa, izlazni napon se regulira na 3.5 V, čime se baterija može u potpunosti iskoristiti. Ovaj čip također ima ugrađene zaštite od prednapona, a jednim otpornikom moguće je i programirati prekostrujnu zaštitu. Također je jednim otpornikom moguće i programirati maksimalnu struju punjenja baterije.
\begin{figure}[hbt]
    \centering
    \includegraphics[width = 10 cm]{Figures/bq24166.png}
    \caption{BQ24166 u QFN kućištu}
    \label{slk:BQ24166}
\end{figure}
Shema baterijskog punjača prikazana je na slici \ref{slk:MB_BATCHG}. Blokadni kondenzatori su postavljeni prema uputama proizvođača, a potrebno je bilo odabrati odgovarajuće otpornike za programiranje prekostrujne zaštite i maksimalne struje punjenja, kao i prikladnu zavojnicu. Kako bi se navedene komponente odabrale na odgovarajuć način, potrebno je znati izlaznu struju iz punjača.
\begin{figure}[hbt]
    \centering
    \includegraphics[width = \textwidth]{Figures/MB_BATCHG.png}
    \caption{Shema baterijskog punjača}
    \label{slk:MB_BATCHG}
\end{figure}

Na izlazu punjača se nalazi linearni regulator napona s niskim padom napona (190 mV na struji od 1.5 A) koji regulira napon na 3.3 V (slika \ref{slk:MB_LDO}). Imajući na umu da je ulazna struja linearnog regulatora otprilike ista kao i izlazna struja, dobiva se izlazna struja baterijskog punjača, što odgavara proračunu iz tablice \ref{tab:MB3V3}, a to je ujedno i struja koju daje baterija. Proizvođač preporuča zavojnice vrijednosti 2.2 $\mu$H za slučaj manje valovitosti struje i manjeg ograničenja na prostor, i 1.5 $\mu$H u slučaju manjeg ograničenja na valovitost i većeg ograničenja na prostor. Ovdje je odabrana vrijednost od 2.2 $\mu$H, a zavojnica je odabrana tako da su struje zasićenja zavojnice i maksimalna struja koju zavojnica može podnijeti veće od dvostruke izlazne struje, što je slučaj najveće valovitosti.

Radi potreba testiranja dodani su pritezni otpornici na stezaljkama USB1, USB2 i USB3. Ovisno o logičkim razinama na tim stezaljkama moguće je ograničiti struju na ulazu za USB. Razne konfiguracije ograničenja struje prikazane su u tablici \ref{tab:USB_ILIM}. Uobičajeno ograničenje u ovom slučaju je 1.5 A.
\begin{table}[htbp]
    \centering
    \caption{Ograničenja struje USB-a \cite{ti:bq24166}}
    \begin{tabular}{|c|c|c|c|c|} \hline
    IUSB3 & IUSB2 & IUSB1 & Ograničenje ulazne struje & Minimalni napon \\
    \hline
    0 & 0 & 0 & 100 mA & 4.28 V \\
    \hline
    0 & 0 & 1 & 500 mA & 4.44 V \\
    \hline
    0 & 1 & 0 & 1.5 A & 4.44 V \\
    \hline
    0 & 1 & 1 & Visoka impedancija & Nikakav \\
    \hline
    1 & 0 & 0 & 150 mA & 4.28 V \\
    \hline
    1 & 0 & 1 & 900 mA & 4.44 V \\
    \hline
    1 & 1 & 0 & 800 mA & 4.44 V \\
    \hline
    1 & 1 & 1 & Visoka impedancija & Nikakav \\
    \hline
    \end{tabular}%
    \label{tab:USB_ILIM}%
\end{table}%

Da bi se dobila maksimalna ulazna struja punjača potrebno je uzeti u obzir najgori mogući slučaj: baterija je odspojena, potrošnja sustava je maksimalna. Unutar punjača se nalazi silazni pretvarač, dakle vrijedi:
\begin{equation}
    P_{IZ}=\eta \cdot P_{UL}
\end{equation}
\begin{equation} \label{eq:BUCK}
    U_{IZ}\cdot I_{IZ}=\eta \cdot U_{UL} \cdot I_{UL}
\end{equation}
gdje je \textit{U\textsubscript{IZ}} i \textit{I\textsubscript{IZ}} izlazni napon, odnosno struja, \textit{U\textsubscript{UL}} i \textit{I\textsubscript{UL}}, ulazni napon, odnosno struja, a $\eta$ učinkovitost. Bacajući pogled u dokumentaciju proizvođača, može se uzeti efikasnost od 90 \% (slika \ref{slk:EFFICIENCY}).
\begin{figure}[hbt]
    \centering
    \includegraphics[width=\textwidth]{Figures/EFFICIENCY.PNG}
    \caption{Ovisnost efikasnosti o izlaznoj struji punjača \cite{ti:bq24166}}
    \label{slk:EFFICIENCY}
\end{figure}
Iz jednadžbe \ref{eq:BUCK} se sada može dobiti izraz za ulaznu struju:
\begin{equation} \label{eq:IN_CURR}
    I_{UL}=\frac{\eta \cdot U_{IZ} \cdot I_{IZ}}{U_{UL}}
\end{equation}
Iz jednadžbe \ref{eq:IN_CURR} je vidljivo da će ulazna struja biti najveća kada je ulazni napon što manji, što u ovom slučaju iznosi 5 V. Za izlazni napon se također uzima najgori slučaj od 4.2 V. Sada se za maksimalnu ulaznu struju punjača uz odspojenu bateriju dobiva iznos od ${I_{UL,BATOFF} = 627.48\quad \textrm{mA}}$.
\begin{figure}[hbt]
    \centering
    \includegraphics[width = 10 cm]{Figures/MB_LDO.png}
    \caption{Linearni regulator napona}
    \label{slk:MB_LDO}
\end{figure}

Imajući na umu da će do maksimalne potrošnje doći rijetko kada, ako uopće, i činjenicu da će sustav imati mogućnost ulaska u način rada mirovanja, može se uzeti kapacitet baterije od 2 Ah, čime se postiže balans trajanja baterije i cijene. U tom slučaju dovoljno je ograničiti punjenje baterije na 1 A, a izračun otpornika je vidljiv na shemi (slika \ref{slk:MB_BATCHG}). Odgovarajuće konstante za izračun otpornika su dobivene iz dokumentacije proizvođača. Sada je isto pomoću jednadžbe \ref{eq:IN_CURR} moguće izračunati ulazni struju; ${I_{UL,BATCHG} = 756\quad \textrm{mA}}$. Ukupna maksimalna ulazna struja je stoga:
\begin{equation}
    I_{UL,MAX}=I_{UL,BATCHG}+I_{UL,BATOFF} = 1.38\quad \textrm{A}
    \label{eq:IN_CURR_MAX}
\end{equation}
Dodajući malo sigurnosne margine, za prekostrujnu zaštitu se onda uzima 1.5 A. Za potrebe testiranja i otklanjanje evenutalnih grešaka na ulaz linearnog regulatora dodan je kratkospojnik.

S obzirom na mnoge opasnosti koje litij-ionske baterije nude, potrebno je dizajnirati prikladnu zaštitu za bateriju. U tu svrhu odabran je BQ29700 proizvođača Texas Instruments, prikazan na slici \ref{slk:BQ29700}. Ovaj integrirani krug ima zaštitu baterije od preniskog i previsokog napona, prejake struje pražnjenja i punjenja i kratkog spoja. Shema sklopa za zaštitu baterije prikazana je na slici \ref{slk:MB_BATPROT}.
\begin{figure}[hbt]
    \centering
    \includegraphics[width=10 cm]{Figures/BQ29700.png}
    \caption{BQ29700}
    \label{slk:BQ29700}
\end{figure}
\begin{figure}[hbt]
    \centering
    \includegraphics[width=10 cm]{Figures/MB_BATPROT.png}
    \caption{Shema sklopa za zaštitu baterije}
    \label{slk:MB_BATPROT}
\end{figure}
Otpornici i kondenzatori su odabrani prema preporukama proizvođača, dok se tranzistori biraju prema potrebama sustava tijekom dizajna \cite{ti:bq29700}. Napon baterije se mjeri preko BAT i VSS stezaljki čipa. U slučaju previsokog napona isključuje se tranzistor Q101, a u slučaju preniskog napona isključuje se tranzistor Q102. Struja se mjeri preko stezaljki V- i VSS, dakle preko otpora tranzistora Q101 i Q102. U slučaju prevelike struje pražnjenja ili kratkog spoja isključuje se tranzistor Q102, a u slučaju prevelike struje punjenja isključuje se tranzistor Q101.

\begin{table}[htbp]
    \centering
    \caption{Pragovi aktiviranja zaštite za bateriju \cite{ti:bq29700}}
    \begin{tabular}{|l|p{1.5cm}|p{1.5cm}|p{2.5cm}|p{2.5cm}|p{2.5cm}|} \hline
    \raggedright
    & Prevelik napon & Premali napon & Prejaka struja punjenja & Prejaka struja pražnjenja & Struja kratkog spoja \\
    \hline
    Prag [V] & 4.275 & 2.800 & -0.100 & 0.100 & 0.5 \\
    \hline
    \end{tabular}%
    \label{tab:BQ29700}%
\end{table}%
U tablici \ref{tab:BQ29700} navedeni su pragovi napona na kojima se aktivira zaštita baterije. Ako su oba tranzistora jednaka, onda za struju kroz tranzistor vrijedi:
\begin{equation} \label{eq:TRANCUR}
    I_Q = \frac{U_{TH}}{2\cdot R_{DS(on)}}
\end{equation}
gdje je \textit{U\textsubscript{TH}} napon praga, a \textit{R\textsubscript{DS(on)}} otpor jednog tranzistora. Iz dokumentacije proizvođača navodi se da tranzistor mora podržavati napon između upravljače elektrode i uvoda u iznosu od $U_{GS}=3.5 \quad \textrm{V}$. Iz grafa ovisnosti \textit{R\textsubscript{DS(on)}} o \textit{U\textsubscript{GS}} prikazanog na slici može se isčitati vrijednost otpora od $R_{DS(on)}=12.5\quad \textrm{m}\Omega$.
\begin{figure}[hbt]
    \centering
    \includegraphics[width=10 cm]{Figures/RDS_OLD.PNG}
    \caption{Graf ovisnosti otpora o naponu između upravljačke elektrode i uvoda tranzistora CSD16406Q3 \cite{ti:csd1640}}
    \label{slk:RDS_OLD}
\end{figure}
Zaštita baterije će se aktivirati u sljedećim uvjetima:
\begin{itemize}
    \item prevelika struja punjenja $I_{OCC}=4\quad \textrm{A}$,
    \item prevelika struja pražnjenja $I_{OCD}=4\quad \textrm{A}$,
    \item struja kratkog spoja $I_{SCD}=20\quad \textrm{A}$.
\end{itemize}
Ovo je dizajn napravljen po uzoru na dokumentaciju proizvođača, međutim, očito je da je ovaj tranzistor loš odabir za ovaj slučaj. Pragovi struja na kojima će se zaštita aktivirati je prevelika, i može doći do oštećenja sklopovlja ili baterije pođe li nešto po zlu. Ispravan način proračuna i odabira tranzistora bit će demonstriran u poglavlju vezan uz izradu narukvice.

\section{USB napajanje}

Evolucijom USB sučelja rasla je snaga koju USB izvor može dati. Tablica \ref{tab:USB} prikazuje snage koje razne verzije USB sučelja mogu dati.
\begin{table}[htbp]
    \centering
    \caption{Maksimalni naponi, struje i snage raznih verzija USB sučelja \cite{ti:usb}}
    \begin{tabular}{|l|c|c|c|} \hline
    Verzija & Maksimalni napon & Maksimalna struja & Maksimalna snaga \\
    \hline
    USB 2.0 & 5 V & 500 mA & 2.5 W \\
    \hline
    USB 3.0 i USB 3.1 & 5 V & 900 mA & 4.5 W \\
    \hline
    USB BC 1.2 & 5 V & 1.5 A & 7.5 W \\
    \hline
    USB Type-C 1.2 & 5 V & 3 A & 15 W \\
    \hline
    USB PD 3.0 & 20 V & 5 A & 100 W \\
    \hline
    \end{tabular}
    \label{tab:USB}
\end{table}
Korisnik obično ne razmišlja na koju verziju sučelja spaja svoj uređaj, moguće je spojiti uređaj na USB 2.0 ili USB PD 3.0. Ako je uređaju potrebno 1.5 A struje, a spojen je na USB 2.0, doći će do pada napona na USB izvoru, te može doći do kvara izvora. Iz tog razloga USB C priključak sadržava CC linije preko kojih uređaj i izvor mogu iskomunicirati potrebnu snagu. Način na koji CC linije funkcioniraju je prikazan na slici \ref{slk:USB_CC_LINES}. Izvor u sebi sadržava pritezne otpornike na napajanje spojene na CC linije, dok uređaj, odnosno ponor na CC linije ima spojene pritezne otpornike na masu. Uređaj promatra CC linije kako bi zaključio kako je kabel orijentiran, dok izvor promatra napon na CC linijama, odnosno naponskom djelilu. Mjereći napon izvor može zaključiti koliku snagu traži uređaj. Dakle, najjednostavniji način za USB napajanje uređaja je spojiti par priteznih otpornika na masu, čime se traži fiksna vrijednost snage. Međutim, uređaj mora imati sposobnost napajanja iz više vrsta izvora, te je potrebno nešto kompleksnije rješenje.
\begin{figure}[htb]
    \centering
    \includegraphics[width=10 cm]{Figures/USB_CC_LINES.png}
    \caption{Prikaz načina rada CC linija \cite{ti:usb}}
    \label{slk:USB_CC_LINES}
\end{figure}
Radi toga je potrebno projektirati sklopovlje koje može ispregovarati snagu koju izvor može dati kako ne bi došlo do neželjenog ponašanja. Postoji niz integriranih krugova koji to mogu napraviti, a ovdje je odabran CYPD3177 tvrtke Cypress Technologies. Shema USB napajanja prikazana je na slici \ref{slk:MB_USB}. Ovaj integrirani krug u sebi ima niz priteznih otpornika na stezaljkama CC1 i CC2 koje su spojene direktno na USB konektor. Struja i napon koju će čip zahtijevati namještaju se naponskim djelilima preko stezaljki ISNK\_COARSE, ISNK\_FINE, VBUS\_MIN i VBUS\_MAX. Vrijednosti otpornika na shemi odabrane su prema savjetima proizvođača \cite{ct:usb}. U ovom slučaju napon koji će se tražiti iznosi 5 V, a struja 1.5 A. Na temelju toga čip postavlja prikladne vrijednosti otpora na CC linije. Ako izvor može dati zahtijevanu snagu čip postavlja stezaljki VBUS\_FET\_EN u visoku razinu, čiji je napon u visokom stanju jednak naponu koji se nalazi na USB priključku. Na taj se način uključuje vanjski P-MOSFET koji prosljeđuje napajanje s USB priključka. Otpori i kondenzatori okolo tranzistora služe s usporavanje rasta napona, čime se izbjegavaju neugodne tranzijentne pojave. Za indikaciju neuspješnog dogovora između izvora i uređaja služi crvena svjetleća dioda D201. Za svrhe testiranja i otklanjanja grešaka predviđeni su kratkospojnici kojima se može odvojiti i preskočiti ovaj dio sklopovlja. U tom slučaju potrebno je zalemiti otpornike R204 i R203 u vrijednosti od 5.1 k$\Omega$, čime se traži snaga od 15 W (5 V, 3 A).
\begin{sidewaysfigure}[htbp!]
    \centering
    \includegraphics[width=1\textwidth]{Figures/MB_USB.png}
    \caption{Shema USB napajanja}
    \label{slk:MB_USB}
\end{sidewaysfigure}
\pagebreak
\section{PCB}
Raspored slojeva tiskane pločice prikazan je na slici \ref{slk:MB_BOARD_STACKUP}. Oba unutarnja sloja su ravnine za uzemljenje kako bi se što bolje očuvao integritet signala na gornjem i donjem sloju. Ova konfiguracija je preuzeta sa web stranice proizvođača pločice JLCPCB.
\begin{figure}[htb]
    \centering
    \includegraphics[width=10 cm]{Figures/MB_BOARD_STACKUP.png}
    \caption{Raspored slojeva PCB-a}
    \label{slk:MB_BOARD_STACKUP}
\end{figure}

Tijekom projektiranja pločice bilo je potrebno posebno obratiti pažnju na smanjenje površine komutacijske petlje kod punjača (slika \ref{slk:MB_SW_NODE}) kojega čine zavojnica L101 te kondenzatori C103 i C107 (slika \ref{slk:MB_BATCHG}) kako bi se smanjile emitirane smetnje. Također je bilo potrebno obratiti pažnju na pozicioniranje antene i ravnine uzemljenja oko nje (slika \ref{slk:MB_ANTENNA}). Ovdje se može uočiti greška u projektiranju jer iako nema uzemljenja ispod antene, on se nalazi okolo antene, što može predstavljati poteškoće u bežičnoj komunikaciji.
\begin{figure}[htb]
    \centering
    \includegraphics[width=6 cm]{Figures/MB_SW_NODE.png}
    \caption{Komutacijska petlja punjača}
    \label{slk:MB_SW_NODE}
\end{figure}
\begin{figure}[htb]
    \centering
    \includegraphics[width=6 cm]{Figures/MB_ANTENNA.png}
    \caption{Antena na PCB-u}
    \label{slk:MB_ANTENNA}
\end{figure}
3D prikaz projektirane pločice može se vidjeti na slici \ref{slk:MB_PCB}.
\begin{figure}[htb]
    \centering
    \includegraphics[width=10 cm]{Figures/MB_PCB.png}
    \caption{3D prikaz pločice}
    \label{slk:MB_PCB}
\end{figure}