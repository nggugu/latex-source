\documentclass[diplomskirad]{fer}
% Dodaj opciju upload za generiranje konačne verzije koja se učitava na FERWeb
% Add the option upload to generate the final version which is uploaded to FERWeb


%\usepackage{blindtext}
\usepackage{rotating}
\usepackage{float}

%--- PODACI O RADU / THESIS INFORMATION ----------------------------------------

% Naslov na engleskom jeziku / Title in English
\title{Wearable measurement system for detection and classification of speech fluency disorders}

% Naslov na hrvatskom jeziku / Title in Croatian
\naslov{Nosivi mjerni sustav za prepoznavanje i klasifikaciju poremećaja tečnosti govora}

% Broj rada / Thesis number
\brojrada{1234}

% Autor / Author
\author{Nikola Gudan}

% Mentor 
\mentor{Prof.\@ Hrvoje Džapo}

% Datum rada na engleskom jeziku / Date in English
\date{June, 2024}

% Datum rada na hrvatskom jeziku / Date in Croatian
\datum{lipanj, 2024.}

%-------------------------------------------------------------------------------


\begin{document}


% Naslovnica se automatski generira / Titlepage is automatically generated
\maketitle


%--- ZADATAK / THESIS ASSIGNMENT -----------------------------------------------

% Zadatak se ubacuje iz vanjske datoteke / Thesis assignment is included from external file
% Upiši ime PDF datoteke preuzete s FERWeb-a / Enter the filename of the PDF downloaded from FERWeb
\zadatak{Nikola Gudan - tekst zadatka - ispravljeno.pdf}


%--- ZAHVALE / ACKNOWLEDGMENT --------------------------------------------------

\begin{zahvale}
  % Ovdje upišite zahvale / Write in the acknowledgment
  Zahvaljujem se svom kolegi i bliskom prijatelju Petru Sušcu na pronalasku interesantne teme diplomskog rada i pružanoj podršci tijekom izrade.
\end{zahvale}


% Odovud započinje numeriranje stranica / Page numbering starts from here
\mainmatter


% Sadržaj se automatski generira / Table of contents is automatically generated
\tableofcontents


%%--- UVOD / INTRODUCTION -------------------------------------------------------
\chapter{Uvod}
\label{pog:uvod}

Zamuckivanje i mucanje se odnosi na poremećaje u ritmu govora u kojima pojedinac zna točno što želi reći, ali tijekom govora ne može pričati radi nehotičnog ponavljajućeg produljenja ili prestanka zvuka \cite{who}. Na mucanje utječe tjeskoba osobe s poremećajem, a povezanost tjeskobe i poremećaja uvjetovane su s vremenom i izloženosti osobe poremećaju.

Mucanje se može kontrolirati logopedskom terapijom. Procjena intenziteta mucanja provodi se tijekom i nakon terapije kako bi se utvrdila učinkovitost terapije. Intenzitet se procjenjuje brojanjem netočnih slogova na uzorku od nekoliko stotina izgovorenih slogova prilikom razgovora između logopeda i pacijenta. Kako ne bi došlo do smetnji u razgovoru između logopeda i pacijenta, razgovor se snima te logoped procjenjuje intenzitet mucanja nakon terapije slušajući snimku jer je teško brojati slogove u stvarnom vremenu. Međutim, na pacijentovu sposobnost govora utječe stres, koji je manje izražen u kontroliranim uvjetima terapije, a više u kaotičnom svijetu van terapije. Radi toga, intenzitet mucanja pokazuje veliku varijabilnost \cite{TICHENOR2015}.

Svrha ovog rada je razviti prototip uređaja koji osoba s poremećajem može, bez većih smetnji, nositi van terapije koji mjeri utjecaj stresa na osobu i snima govor osobe. Uređaj obrađuje prikupljene podatke i mjeri intenzitet mucanja s obzirom na razinu stresa korisnika.

Sustav prikuplja zvukovne podatke i fiziološke signale na temelju kojih se određuje intenzitet mucanja i razina stresa. Fiziološki signali prikupljaju se radi određivanja razine stresa korisnika i moraju se mjeriti neinvazivnom metodom radi udobnosti korisnika. Obrađeni podaci pohranjuju se lokalno na uređaj kako bi logoped kasnije te podatke mogao preuzeti i analizirati ih. Iako postoje sustavi koji mjere razne biomedicinske parametre, poput fitnes narukvica i pametnih satova, za sada ne postoji sustav koji objedinjuje mjerenje fizioloških signala s govornim podacima u svrhu određivanja intenziteta mucanja.

Radi zahtjeva na nosivost potrebno je napraviti prikladni sustav napajanja prateći trendove u nosivim uređajima, poput korištenja litij-ionskih baterija, mogućnost brzog punjenja, kompatibilnost s USB-C priključkom i male dimenzije uređaja.

U radu su opisani razlozi za odabir ključnih elektroničkih komponenti, proračun oscilatora, proračun potrošnje energije, metodologija mjerenja biomedicinskih signala, prikupljanja zvučnih zapisa te postupak projektiranja, izrade i ispitivanja realiziranog uređaja.
%-------------------------------------------------------------------------------
%--- UVOD / INTRODUCTION -------------------------------------------------------
\chapter{Uvod}
\label{pog:uvod}

Zamuckivanje i mucanje se odnosi na poremećaje u ritmu govora u kojima pojedinac zna točno što želi reći, ali tijekom govora ne može pričati radi nehotičnog ponavljajućeg produljenja ili prestanka zvuka \cite{who}. Na mucanje utječe tjeskoba osobe s poremećajem, a povezanost tjeskobe i poremećaja uvjetovane su s vremenom i izloženosti osobe poremećaju.

Mucanje se može kontrolirati logopedskom terapijom. Procjena intenziteta mucanja provodi se tijekom i nakon terapije kako bi se utvrdila učinkovitost terapije. Intenzitet se procjenjuje brojanjem netočnih slogova na uzorku od nekoliko stotina izgovorenih slogova prilikom razgovora između logopeda i pacijenta. Kako ne bi došlo do smetnji u razgovoru između logopeda i pacijenta, razgovor se snima te logoped procjenjuje intenzitet mucanja nakon terapije slušajući snimku jer je teško brojati slogove u stvarnom vremenu. Međutim, na pacijentovu sposobnost govora utječe stres, koji je manje izražen u kontroliranim uvjetima terapije, a više u kaotičnom svijetu van terapije. Radi toga, intenzitet mucanja pokazuje veliku varijabilnost \cite{TICHENOR2015}.

Svrha ovog rada je razviti prototip uređaja koji osoba s poremećajem može, bez većih smetnji, nositi van terapije koji mjeri utjecaj stresa na osobu i snima govor osobe. Uređaj obrađuje prikupljene podatke i mjeri intenzitet mucanja s obzirom na razinu stresa korisnika.

Sustav prikuplja zvukovne podatke i fiziološke signale na temelju kojih se određuje intenzitet mucanja i razina stresa. Fiziološki signali prikupljaju se radi određivanja razine stresa korisnika i moraju se mjeriti neinvazivnom metodom radi udobnosti korisnika. Obrađeni podaci pohranjuju se lokalno na uređaj kako bi logoped kasnije te podatke mogao preuzeti i analizirati ih. Iako postoje sustavi koji mjere razne biomedicinske parametre, poput fitnes narukvica i pametnih satova, za sada ne postoji sustav koji objedinjuje mjerenje fizioloških signala s govornim podacima u svrhu određivanja intenziteta mucanja.

Radi zahtjeva na nosivost potrebno je napraviti prikladni sustav napajanja prateći trendove u nosivim uređajima, poput korištenja litij-ionskih baterija, mogućnost brzog punjenja, kompatibilnost s USB-C priključkom i male dimenzije uređaja.

U radu su opisani razlozi za odabir ključnih elektroničkih komponenti, proračun oscilatora, proračun potrošnje energije, metodologija mjerenja biomedicinskih signala, prikupljanja zvučnih zapisa te postupak projektiranja, izrade i ispitivanja realiziranog uređaja.

\chapter{Građa uređaja}
\label{pog:structure}

Sustav se sastoji od dva uređaja koji zajedno rade u prikupljanju i obradi podataka. Središnji uređaj služi za snimanje, obradu i pohranu glasovnih podataka i obradu i pohranu biomedicinskih parametara. Za snimanje biomedicinskih parametara koristi se narukvica. Središnji uređaj i narukvica razmjenjuju podatke i naredbe putem Bluetooth protokola. Prikupljeni podaci se obrađuju pomoću neuralnih mreža kako bi se utvrdio intenzitet mucanja. Blok dijagram sustava prikazan je na slici \ref{slk:BD_MAIN}.
\begin{figure}[htb]
    \centering
    \includegraphics[width=\textwidth]{Figures/block_diagram.drawio.png}
    \caption{Blok dijagram sustava}
    \label{slk:BD_MAIN}
\end{figure}

Na središnjem uređaju nalazi se mikrokontroler (MCU) za obradu podataka, MEMS mikrofon za snimanje govora korisnika, SD kartica za lokalnu pohranu podataka i čip za praćenje vremena (RTC) za usklađivanje govornih podataka i biomedicinskih podataka. Podaci će se pohranjivati lokalno na uređaj kako bi logoped mogao kasnije preuzeti podatke te ih analizirati.

Narukvica mjeri brzinu otkucaja srca putem fotopletizmografskog senzora (PPG) i impedanciju kože s pomoću instrumentacijskog pojačala. Promjena impedancije kože je dobar pokazatelj stresa u korisnika \cite{edr}, a osobe s poremećajem tečnosti govora pokazuju značajno smanjenje brzine otkucaja srca u stresnim situacijama u odnosu na osobe bez takvih poremećaja \cite{ALM2004123}.

Oba uređaja sadržavaju sustav za bežičnu komunikaciju, baterijsko napajanje i mogućnost punjenja. Radi velike raširenosti koristit će se litij-ionska baterija i mogućnost punjenja putem USB C sučelja.

Zbog zahtjeva na nosivost uređaja, veličine tiskanih pločica (engl. \textit{Printed Circuit Board}, PCB), imaju ograničenje na fizičku veličinu, međutim, s obzirom na to da se radi o prototipu, uređaji će sadržavati testne točke i dovoljno velike komponente kako bi eventualna prerada pločice bila lakša. S obzirom na ta dva zahtjeva potrebno je napraviti pločicu koja će biti kompromis između ta dva zahtjeva.

Prvo će biti izrađen središnji sustav kako bi se ispitala mogućnost prikupljanja glasovnih podataka i sustav za napajanje na temelju čega će kasnije biti izrađena narukvica kako bi se upotpunile tražene funkcionalnosti sustava.

\chapter{Glavna ploča}
\label{pog:mainboard}

Sustav se sastoji od dva uređaja koji rade u simbiozi. Glavna ploča služi za snimanje, obradu i pohranu glasovnih podataka i obradu i pohranu biomedicinskih parametara. Za snimanje biomedicinskih parametara koristi se narukvica. U ovom poglavlju se opisuje glavna ploča.

Zahtjevi na glavnu ploču su sljedeći:
\begin{itemize}
    \item mikrokontroler \engl{Microcontroler Unit, MCU}, dovoljno moćan za pokretanje neuralnih mreža i obradu podataka
    \item konektor za SD karticu
    \item bežična komunikacija putem Wi-Fi ili Bluetooth sučelja
    \item praćenje vremena putem RTC-a
    \item mikrofon za prikupljanje govora korisnika
    \item sučelja za testiranje i prženje koda na mikrokontroler
    \item napajanje i punjenje baterije preko USB C priključka
    \item baterijsko napajanje putem litij-ionske baterije
\end{itemize}
U daljnjem tekstu ovog poglavlja opisane su odabrane komponente, kao i razlog njihova odabira, način, razlozi i proračuni dizajna pojedinih podsustava, te dizajn, proizvodnja i testiranje PCB-a.

\section{Mikrokontroler}
Za mikrokontroler odabran je STM32F746VG baziran na Cortex-M7 arhitekturi koji integrira funkcionalnosti digitalne obrade signala, bogat sa svim potrebnim periferijama za integriranje s ostatkom sustava i dovoljno procesorske snage za obavljanje zadanog zadatka. Također, programska potpora je razvijena na razvojnom sustavu BLABLABLA, pa je ovaj mikrokontroler odabran radi lakšeg razvoja cjelokupnog sustava. Shema napajanja mikrokontrolera prikazana je na slici \ref{slk:MCU_PS}, a shema spajanja mikronotrolera sa ostatkom sustava prikazana je na slici \ref{slk:MCU_PE}.

\begin{figure}[hbt]
    \centering
    \includegraphics[width=\textwidth]{Figures/MCU_02.png}
    \caption{Shema napajanja mikrokontrolera}
    \label{slk:MCU_PS}
\end{figure}

Shema napajanja napravljena je prema uputama proizvođača \cite{stmicroelectronics:an4661}. S obzirom na to da na ovoj ploči nema analognih signala, nije potrebno raditi analogno-digitalnu pretvorbu, pa su stezaljke za napajanje analognog dijela mikrokontrolera spojene sa stezaljkama za napajanje digitalnog dijela. Također, nije potrebna precizna naponska referenca, a baterijskim napajanjem će upravljati vanjski čip, pa su te dvije stezaljke spojene na napajanje od +3.3V.

\begin{figure}[hbt]
    \centering
    \includegraphics[width=\textwidth]{Figures/MCU_01.png}
    \caption{Shema periferije mikrokontrolera}
    \label{slk:MCU_PE}
\end{figure}

Prilikom dizajna 

\chapter{Narukvica}
\label{pog:bracelet}
Svrha narukvice je prikupljanje biomedicinskih parmetara korisnika i njihovo slanje na obradu na glavnoj ploči. Biomedicinski parametri koji će se promatrati su brzina otkucaja srca putem fotopletizmografskog senzora (PPG) i impedancija kože, odnosno elektrodermalna aktivnost. Promjena elektrodermalna aktivnost je dobar pokazatelj stresa u korisnika \cite{edr}, a osobe sa poremećajem tečnosti govora pokazuju značajno smanjenje brzine otkucaja srca u stresnim situacijama u odnosu na osobe bez takvih poremećaja \cite{ALM2004123}.

Što se tiče zahtjeva na napajanje narukvice, situacija je ista kao i kod glavne ploče, uz drugačiju potrošnju. Tako da izrada pločice za narukvicu predstavlja mogućnost ispravljanja grešaka nastalih tijekom dizajna napajanja glavne ploče. Narukvica također mora imati mogućnost bežične komunikacije. S obzirom na ograničenje veličine ploče maknuti su kratkospojnici i testne točke.

\section{Bežična komunikacija}
Shema bežične komunikacije na narukvici (slika \ref{slk:BR_WIRELESS}) je veoma slična onoj na glavnoj ploči (slika \ref{slk:WIFI}), uz nedostatak kratkospojnika, dodatak signala za upravljanje I\textsubscript{2}C sučeljem i korištenje analogno-digitalnog pretvornika za mjerenje impadancije kože. Još jedna promjena dolazi u obliku programiranja preko UART-a. S obzirom na probleme tijekom programiranja glavne ploče dodani su signali DTR i CTS kako bi se BOOT i EN stezaljke ESP mikrokontrolera mogle programski upravljati.
\begin{sidewaysfigure}[htbp]
    \centering
    \includegraphics[width=1\textwidth]{Figures/BR_WIRELESS.png}
    \caption{Shema bežične komunikacije narukvice}
    \label{slk:BR_WIRELESS}
\end{sidewaysfigure}

\newpage
\section{Fotopletizmografski senzor}

Za mjerenje brzine otkucaja srca koristi se PPG senzor MAX30101 tvrtke Analog Devices (slika \ref{slk:MAX30101}). Ovaj snezor u sebi ima crvenu, zelenu i infracrvenu svjetleću diodu i fotosenzor, upravljačko sklopovlje za diode, te komunicira preko I\textsuperscript{2}C sučelja. Kao što je vidljivo na shemi na slici \ref{slk:PPG} ovaj senzor je veoma jednostavan za implementaciju uz svega par par priteznih otpornika i blokadnih kondenzatora. Jedina komplikacija dolazi u obliku napajanja od 5 V, koje je potrebno jer je pad napona na zelenoj svjetlećoj diodi specificiran na 3.3 V.
\begin{figure}[htb]
    \centering
    \includegraphics[width=6 cm]{Figures/MAX30101.JPG}
    \caption{MAX30101 PPG senzor}
    \label{slk:MAX30101}
\end{figure}
\begin{figure}[htb]
    \centering
    \includegraphics[width=\textwidth]{Figures/PPG.png}
    \caption{Shema PPG senzora}
    \label{slk:PPG}
\end{figure}

\section{Impedancija kože}
Impedancija kože mjerit će se pomoću instrumentacijskog pojačala. Shema mjernog kruga prikazana je na slici \ref{slk:EDR}. Odabrano je instrumentacijsko pojačalo AD8226 tvrtke Analog Devices zbog svog velikog ulaznog otpora, malog šuma i dobrog potiskivanja zajedničkih smetnji \cite{ad:ad8226}. Napajanje pojačala je filtrirano pasivnom mrežom kako ne bi došlo do smetnji od digitalnog dijela sklopovlja.
\begin{figure}[htb]
    \centering
    \includegraphics[width=\textwidth]{Figures/EDR.png}
    \caption{Shema mjernog kruga za impedanciju kože}
    \label{slk:EDR}
\end{figure}
Za mjerenje impedancije koristi se referentni napon od 0.5 V, a mjerenje impedancije se temelji na mjerenju napona na naponskom djelilu na stezaljci -IN pojačala. Kožu predstavlja donji otpornik u naponskom djelilu, te razlika između tog napona i referentnog napona pojačava:
\begin{equation} \label{eq:EDR}
    U_{IZ}=A\cdot U_{REF}\cdot \frac{R_{403}}{R_{403}+R_{skin}}
\end{equation}
Impedancija kože mjeri se u stotinama kilooma (maksimalno cca. $250\quad \textrm{k}\Omega$ \cite{rskin}), tako da je vrijednost gornjeg otpornika $200\quad \textrm{k}\Omega$. Pojačanje iznosi 2 i namješta se preko otpornika R402.

Za svrhe lakšeg prototipiranja koristit će se samoljepljive elektrode (slika \ref{slk:ELECTRODE}) koje se montiraju na kabel prikazan na slici \ref{slk:CABLE}. 
\begin{figure}[htb]
    \centering
    \includegraphics[width=6 cm]{Figures/ELECTRODE-BOTTOM.jpg}
    \caption{Samoljepljiva elektroda}
    \label{slk:ELECTRODE}
\end{figure}
\begin{figure}
    \centering
    \includegraphics[width=6 cm]{Figures/CABLE.jpg}
    \caption{Kabel za elektrode}
    \label{slk:CABLE}
\end{figure}
Ovaj kabel se spaja na narukvicu putem 3.5 mm audio priključka.

\section{Napajanje}
\subsection{Proračun potrošnje}

Proračun potrošnje za narukvicu je bio puno jednostavniji od proračuna za glavnu ploču. U ovom slučaju, najveći potrošač je i dalje sustav za bežičnu komunikaciju, međutim on je efektivno i jedini potrošać na 3.3 V jer je potrošnja instrumentacijskog pojačala izrazito mala, maksimalno $20\quad \mu \textrm{A}$ \cite{ad:ad8226}. Za potrošnju sustava bežične komunikacije uzima se vrijednost prikazana u tablici \ref{tab:MB3V3}. Na napajanju od 5 V jedini potrošač je PPG senzor i njegova potrošnja u najgorem slučaju iznosi 50 mA, a na napajanju od 1.8 V senzor troši maksimalno 1.1 mA \cite{ad:max30101}.

\subsection{Napajanja od 3.3 V i 1.8 V}
\label{subsec:BR_VDD}
Shema napajanja od 3.3 V i 1.8 V prikazana je na slici \ref{slk:BR_VDD}. U oba slučaja koristi se LDO AP2112K proizvođača Diodes Incorporated. Ovaj LDO je odabran radi svoje male veličine u SOT-25 s obzirom na ograničenje veličine PCB-a. 
\begin{figure}[htb]
    \centering
    \includegraphics[width=6 cm]{Figures/BR_VDD.png}
    \caption{Napajanje od 3.3 V i 1.8 V za narukvicu}
    \label{slk:BR_VDD}
\end{figure}

\subsection{Referentni napon}

Shema izvora referentnog napona prikazana je na slici \ref{slk:BR_VREF}. Koristi se ADR130 referenca tvrtke Analog Devices. Iznos referentnog napona se može namjestiti na 1 V ili 0.5 V, a veličina kućišta je ista kao i kod linearnih regulatora prikazanih u dijelu \ref{subsec:BR_VDD}.
\begin{figure}[htb]
    \centering
    \includegraphics[width=10 cm]{Figures/BR_VREF.png}
    \caption{Referentni izvor napona od 0.5 V}
    \label{slk:BR_VREF}
\end{figure}

\subsection{Napajanje od 5 V}

Za napajanje od 5 V bilo je potrebno dizajnirati uzlazni prekidački regulator. Shema regulatora prikazana je na slici \ref{slk:BR_BOOST}. Odabran je TLV61220 proizvođača Texas Instruments jer je idealan za napajanje sa baterije. Regulator može raditi na ulaznom naponu od 0.7 V do 5.5 V i potrebno je malo komponenata za rad \cite{ti:tlv61220}. Također je pogodan radi svoje male veličine u SOT-23 kućištu.
\begin{figure}[htb]
    \centering
    \includegraphics[width=13 cm]{Figures/BR_BOOST.png}
    \caption{Uzlazni prekidački regulator}
    \label{slk:BR_BOOST}
\end{figure}
Zavojnica je odabrana prema preporukama proizvođača, a naponsko djelilo je proračunato imajući na umu da donji otpornik ne bi trebao biti veći od $500\quad \textrm{k}\Omega$ kako bi vrijednost struje koja teče u FB stezaljku bila što bliže $0.01 \quad \mu \textrm{A}$ \cite{ti:tlv61220}.
Efikasnost za izlazne struje od 1 mA do 50 mA je skoro ista na ulaznom naponu u rasponu baterije i može se uzeti efikasnost od 90 \% \ref{slk:BOOST_EFF}. Uz minimalni ulazni napond od 3V, potrošnja, prema jednadžbi \ref{eq:IN_CURR} iznosi 75 mA.
\begin{figure}[htb]
    \centering
    \includegraphics[width=8 cm]{Figures/BOOST_EFF.png}
    \caption{Efikasnost regulatora \cite{ti:tlv61220}}
    \label{slk:BOOST_EFF}
\end{figure}

\section{Baterija i punjač baterije}

Uzevši u obzir potršnju svih podsustava ukupna struja koju punjač mora moći dati je 426.12 mA. Uz struju punjenja baterije od 1 A i uzevši u obzir jednadžbe \ref{eq:IN_CURR} i \ref{eq:IN_CURR_MAX} ukupna struja koju USB sučelje mora moći dati iznosi 1.08 A, dakle uzet će se ograničenje na ulaznu struju od 1.5 A. Uvijeti su, dakle, veoma slični onima kao kod glavne ploče.

\subsection{Punjač baterije}
Shema punjača narukvice na slici \ref{slk:BR_BATCHG} veoma je slična shemi sa slike \ref{slk:MB_BATCHG}. Promijenjena je svjetleća dioda za indikaciju punjenja u žutu, kako bi se jasnije mogla razlikovati indikacija između indikacije punjenja i indikacije statusa dobrog napajanja. Također su dodani veći otpornici u seriju sa diodama jer je svjetlina bila prevelika.

Još jedna razlika je u priteznim otpornicima na konfiguracijskim linijama za ograničenje struje USB-a, ovdje je ograničenje uvijek 1.5 A jer nema potrebe za drugačijim postavkama i postoji ograničenje na veličinu pločice. S obzirom na ograničenje na veličinu pločice ovdje je odabrana zavojnica od 1.5 $\mu \textrm{H}$.
\begin{figure}[htb]
    \centering
    \includegraphics[width=\textwidth]{Figures/BR_BATCHG.png}
    \caption{Shema punjača baterije}
    \label{slk:BR_BATCHG}
\end{figure}

Zadnja razlika je vrlo subtilna, ali veoma ključna za ispravan rad punjača. Pogledom na shemu na slici \ref{slk:MB_BATCHG} vidljivo je da stezaljka TS nije nigdje spojena, odnosno ,,pluta''. Ova stezaljka služi za mjerenje temperature baterije tijekom punjenja. Ako je temperatura prevelika ili premala punjenje se zaustavlja. Shema mjerenja prikazana je na slici \ref{slk:BATCHG_TS}. Mjeri se napon na naponskom djelilu kojega čine otpornici i NTC termistor. Naponi na kojima se zaštita aktivira iznose 30\% i 60\% napona na stezaljci DRV, dakle 1.56 V i 3.12 V. S obzirom da stezaljka TS ,,pluta'', napon na stezaljci je manji od donjega praga i punjenje ne radi. Kako bi se mjerenje temperature onemogućilo napon na TS stezaljci mora biti veći od 70\% napona na stezaljci DRV. Iz tog razloga proizvođač preporuča kratko spajanje stezaljka TS i DRV kako bi punjenje cijelo vrijeme bilo omogućeno.
\begin{figure}[htb]
    \centering
    \includegraphics[width=\textwidth]{Figures/BATCHG_TS.png}
    \caption{Mjerenje temprature senzora}
    \label{slk:BATCHG_TS}
\end{figure}

\subsection{Baterijska zaštita}


\chapter{Ispitivanje razvijenog sklopovskog rješenja}
U postupku provjere realiziranog rješenja ispitani su sustavi za napajanje, žična komunikacija između dijelova sustava, bežična komunikacija, upisivanje korisničkog programa u Flash memoriju mikrokontrolera, snimanje zvuka, pohrana podataka na SD karticu, RTC, komunikacija s PPG senzorom i mjerenje impedancije kože.

U svim testovima korišten je multimetar za mjerenje napona i osciloskop za promatranje signala u žičnim komunikacijama između sustava. Za ispitivanje napajanja korišteni su USB strujni adapter koji podržava mogućnost brzog punjenja, različite litij-ionske baterije u 18650 kućištu i USB ispitivač UT658DUAL tvrtke Changan UNI-T, prikazan na slici \ref{slk:UT658DUAL}. Ovaj uređaj se može spojiti između USB izvora i uređaja te pokazuje razinu napona, jakost struje koju izvor daje, količinu potrošenog naboja u mAh i vrijeme trajanja mjerenja.
\begin{figure}[htb]
    \centering
    \includegraphics[width=6 cm]{Figures/UT658DUAL.png}
    \caption{USB ispitivač UT658DUAL \cite{tester}}
    \label{slk:UT658DUAL}
\end{figure}
\section{Središnji uređaj}
Na slikama \ref{slk:MB_TEST_01} i \ref{slk:MB_TEST_02} prikazano je ispitivanje središnjeg uređaja. Umjesto priključka za bateriju, priključen je držač za standardnu 18650 bateriju radi lakšeg testiranja. Naime, baterija se često vadi i stavlja tijekom testiranja, a odabrani priključak predviđen je da se rijetko odspaja pa ga je stoga i teško odspajati veći broj puta tijekom ispitivanja.
\begin{figure}[htb]
    \centering
    \includegraphics[width=10 cm]{Figures/MB_TEST_01.jpg}
    \caption{Ispitivanje pločice središnjeg uređaja}
    \label{slk:MB_TEST_01}
\end{figure}
\begin{figure}[htb]
    \centering
    \includegraphics[width=10 cm]{Figures/MB_TEST_02.jpg}
    \caption{Postav ispitivanja pločice središnjeg uređaja}
    \label{slk:MB_TEST_02}
\end{figure}

U postupku ispitivanja otkrivena je pogreška u dizajnu USB napajanja. Naime, kada se spoji USB izvor napajanja, bez obzira podržava li izvor zatraženu snagu ili ne, napajanje s USB-a se ne prosljeđuje prema ostatku sustava. Ustanovljeno je da je problem u tranzistorskoj sklopci (slika \ref{slk:MB_USB}). Naime, kada integrirani sklop pokuša uključiti tranzistore, na prvom tranzistoru dolazi do pada napona na diodi koja se nalazi unutar tranzistora pa se na uvodu drugog tranzistora nalazi napon napajanja umanjen za napon provoda diode (između 0.5 V i 1.2 V \cite{di:dmp3098}). Integrirani sklop je preko VDC\_OUT priključka detektirao previsok pad napona i onemogućio napajanje s USB-a. Radi toga su napravljene izmjene u USB napajanju kod narukvice, koje su vidljive na slici \ref{slk:BR_USB}.

Kada baterija nije bila spojena, nije bilo moguće uključiti USB napajanje iz razloga objašnjenog u potpoglavlju \ref{sec:BR_USB}. Ukratko, integrirani sklop nije mogao postaviti odgovarajuće otpornike na CC linije pa se nije mogla zatražiti nikakva snaga od USB izvora.

Tijekom testiranja punjača utvrđeno je da punjač može bez problema proslijediti napon baterije na izlaz. Međutim, kada se priključio vanjski napon, bilo na USB ili preko priključka za laboratorijski izvor napona u svrhu punjenja baterije, punjač više nije radio. Bilo je vidljivo treperenje svjetlećih dioda za indikaciju ispravnog napajanja i punjenja frekvencijom 1 Hz. Naime, dolazi do aktiviranja temperaturne zaštite na način opisan u potpoglavlju \ref{sec:BR_BATCHG}.

Paralelno s postupkom izrade tiskane pločice, razvijena je programska podrška za središnji uređaj. S obzirom da veći dio podsustava za napajanje, izuzev baterije, nije radio, tijekom testiranja programske podrške, a samim time i digitalnog dijela sustava, uređaj se napajao putem programatora, koji je bio priključen na pločicu cijelo vrijeme tijekom testiranja programske podrške. Utvrđeno je da mikrofon normalno snima govor, da mikrokontroler komunicira sa svim podsustavima i također da bez poteškoća obrađuje i sprema podatke na SD karticu. Utvrđeno je nadalje da se bežični podsustav uspješno može programirati te da RTC mjeri vrijeme uz napajanje s litijske baterije, čime je potvrđeno da digitalni dio sustava u potpunosti radi kako je zamišljeno.

\section{Narukvica}

\begin{figure}[htb]
    \centering
    \includegraphics[width=10 cm]{Figures/BR_TEST_02.jpg}
    \caption{Postav ispitivanja pločice narukvice}
    \label{slk:BR_TEST_01}
\end{figure}
\begin{figure}[htb]
    \centering
    \includegraphics[width=10 cm]{Figures/BR_TEST_01.jpg}
    \caption{Ispitivanje pločice narukvice}
    \label{slk:BR_TEST_02}
\end{figure}

Na temelju ispitivanja središnjeg uređaja, napravljene su izmjene u dizajnu napajanja koje su implementirane na pločici narukvice, kako je opisano u poglavlju \ref{pog:bracelet}. Ispitivanje pločice narukvice prikazano je na slikama \ref{slk:BR_TEST_01} i \ref{slk:BR_TEST_02}.

Tijekom ispitivanja primijećena je pogreška u dizajnu baterijskog napajanja. Ime mreže koja je spojena na pozitivan terminal baterije (slika \ref{slk:BR_BATPROT}) i ime mreže koja je spojena na ulaz za bateriju na punjaču (slika \ref{slk:BR_BATCHG}) su različiti. Radi toga baterija, nakon zaštite, nije pogreškom bila nigdje spojena pa je bilo potrebno te dvije mreže kratko spojiti žicom naknadno, čime je napajanje proradilo.

Podsustav za USB napajanje sada može upravljati napajanjem i kada je na uređaj spojeno samo USB napajanje. Također, narukvica se sada može napajati putem USB-a i kada baterija nije spojena. Baterijski punjač može proslijediti napajanje baterije ili može regulirati napajanje USB-a.
\begin{figure}[htb]
    \centering
    \includegraphics[width=10 cm]{Figures/BR_TEST_04.jpg}
    \caption{Brzo punjenje, punjenje konstantnom strujom}
    \label{slk:BR_TEST_04}
\end{figure}
\begin{figure}[htb]
    \centering
    \includegraphics[width=10 cm]{Figures/BR_TEST_03.jpg}
    \caption{Brzo punjenje, punjenje konstantnim naponom}
    \label{slk:BR_TEST_03}
\end{figure}
Punjenje baterije konstantnom strujom i konstantnim naponom prikazano je na slikama \ref{slk:BR_TEST_04} i \ref{slk:BR_TEST_03}. Može se vidjeti da sustav ima napajanje i tijekom punjenja, dakle punjač također radi ispravno. Prekidački i linearni regulatori provjereni su multimetrom te na svojim izlazima imaju napon za koji su projektirani.

Tijekom ispitivanja mjernog lanca za EDR primijećeno je da je priključak za elektrode loše kvalitete jer se je periodički gubio kontakt. Radi toga je taj priključak odlemljen, a žice za EDR su izravno zalemljene na tiskanu pločicu. Dodatno je uočen problem osciliranja pojačala, što je riješeno preuzorkovanjem i usrednjavanjem više mjerenja s ADC-a. Nakon toga su dobivena stabilna mjerenja koja su odgovarala očekivanim rezultatima.

Ispravnost PPG senzora nije nažalost mogla biti verificirana jer je tijekom postupka montaže senzor bio oštećen. Radi toga je na sustav spojen modul sa senzorom MAX30100, koji se od MAX30101 razlikuje u tom što nema zelenu svjetleću diodu. Prikaz načina spoja modula na pločicu prikazan je na slici \ref{slk:BR_TEST_05}. Na ovaj način ispitana je razvijena programska podrška za PPG mjerenje. S obzirom da se radi o modulu, sklopovsko rješenje je očito ispravno.
\begin{figure}
    \centering
    \includegraphics[width=10 cm]{Figures/BR_TEST_05.jpg}
    \caption{Spoj modula sa MAX30100 na pločicu}
    \label{slk:BR_TEST_05}
\end{figure}

ESP32 modul koji se nalazi na pločici programiran je putem USB sučelja i pokazano je da se uspješno mogao programirati, čime je verificirana njegova sklopovska ispravnost.


%--- ZAKLJUČAK / CONCLUSION ----------------------------------------------------
\chapter{Zaključak}
\label{pog:zakljucak}

S obzirom na zahtjeve uređaja, sklopovlje prototipa uređaja razvijeno u ovome radu bi se moglo smatrati skoro gotovim. Jedini dio sustava s kojim je došlo do poteškoća, gledajući sa strane dizajna, jest napajanje, a i kasnije su spomenute poteškoće otklonjene ispravnim dizajnom. Drugi problem je bio dvostrana montaža pločice. Međutim, imajući na umu da se u ovom radu pločica montirala ručno, moguće je zaključiti da do ovakvih problema neće doći u uvjetima profesionalne proizvodnje. Sljedeći korak bi bio ovaj prototip unaprijediti tako da bude što manje invazivan na udobnost korisnika kako bi se u potpunosti ispunili zahtjevi na uređaj.

Za početak, potrebno je dizajnirati elektrode za mjerenje impedancije kože. U sadašnjem stadiju, impedancija se mjeri preko samoljepljivih elektroda spojenih na uređaj s pomoću veoma duge žice. Očito je da je takva metoda u potpunosti neprihvatljiva za korisnika koji bi ovaj uređaj trebao nositi cijeli dan uokolo dok obavlja razne zadatke. Cilj je napraviti narukvicu koja će elektrode imati izložene na kućištu narukvice, koja će tako ostvarivati kontakt s kožom. Elektrode bi se onda montirale na donju stranu pločice, kao što je zamišljeno s PPG senzorom.

Nadalje, potrebno je smanjiti fizičke dimenzije pločica. To će se ostvariti uklanjanjem testnih točaka i kratkospojnika koji zauzimaju veliku količinu prostora, kako radi svoje veličine, tako radi svojih velikih međusobnih razmaka na pločici koji su bili potrebni za ispravno i jednostavno testiranje. Još jedan značajniji način na koji će se smanjiti veličina pločice jest korištenje manjih komponenata. To su većinom pasivne komponente, koje su trenutačno u 0603 kućištu, a odokativnom procjenom, korištenje kućišta 0402 bi već smanjile dimenzije pločica za više od pola. Tu je također priključak za SD karticu, koji zauzima površinu veličine sustava za bežičnu komunikaciju. Korištenjem microSD kartice ta površina će se smanjiti četiri puta. Također bi bilo moguće smanjiti sustav za bežičnu komunikaciju tako da se dizajnira vlastiti sustav od diskretnih komponenata, za razliku od modula koji se trenutačno nalazi na pločici.

Također je potrebno promijeniti mikrofon. Naime, mikrofon se sada nalazi montiran direktno na pločicu središnjeg sustava. To je nepraktično jer bi korisnik morao nositi kutiju značajne veličine montiranu negdje blizu glave radi boljeg primitka zvuka. To bi se moglo riješiti klasičnim mikrofonom u bubici, koja je dizajnirana upravo tako da se zakači što bliže glavi, a da pritom što manje smeta. Taj bi se mikrofon onda mogao spajati na središnji sustav žicom koji se nalazi negdje blizu struka korisnika ili gdje god ga korisnik želi montirati.

Iako je filtriranje signala sa sustava za mjerenje impedancije kože izvedeno softverski, bilo bi bolje to učiniti sklopovski. Koristi se manje procesorskih resursa i time se više vremena prepušta mjerenju. To se može napraviti klasičnim niskopropusnim filtrom prvoga reda. Impedancija kože se vrlo sporo mijenja, pa će takav filtar biti dovoljno dobar za ovu situaciju.

Naravno, nema smisla da korisnik uokolo hoda s golim pločicama, iz očitih razloga. Potrebno je dizajnirati kućište i način na koji će korisnik uređaj staviti na sebe, a da mu on što manje smeta dok ga nosi.

%--- LITERATURA / REFERENCES ---------------------------------------------------

% Literatura se automatski generira iz zadane .bib datoteke / References are automatically generated from the supplied .bib file
% Upiši ime BibTeX datoteke bez .bib nastavka / Enter the name of the BibTeX file without .bib extension
\bibliography{literatura}



%--- SAŽETAK / ABSTRACT --------------------------------------------------------

% Sažetak na hrvatskom
\begin{sazetak}
Projektiranje i izrada sklopovlja za nosivi mjerni sustav za prepoznavanje i klasifikaciju poremećaja tečnosti govora. Sustav se sastoji od dva nosiva uređaja. Jedan je za snimanje glasovnih podataka s MEMS mikrofonom, obradu podataka i njihovu pohranu na SD karticu. Drugi je za snimanje biomedicinskih signala (PPG i EDR) u obliku nosive narukvice. Oba sustava imaju funkcije bežične komunikacije preko Wi-Fi ili Bluetooth sučelja, napajanje s litij-ionskom baterijom te napajanje i punjenje preko USB C sučelja. Opisani su dizajn sustava, projektiranje tiskane pločice i funkcionalno ispitivanje sastavljenih uređaja.
\end{sazetak}

\begin{kljucnerijeci}
MEMS mikrofon, SD kartica, PPG, EDR, Wi-Fi, Bluetooth, litij-ionska baterija, USB C sučelje
\end{kljucnerijeci}


% Abstract in English
\begin{abstract}
  Design and production of hardware for a wearable measurement system for the recognition and classification of speech fluency disorders. The system consists of two wearable devices. One is for recording voice data with a MEMS microphone, processing the data, and storing it on an SD card. The other is for recording biomedical signals (PPG and EDR) in the form of a wearable wristband. Both systems have wireless communication functions via Wi-Fi or Bluetooth interfaces, are powered by a lithium-ion battery, and can be powered and charged via a USB-C interface. The design of the system, the printed circuit board (PCB) design, and the functional testing of the assembled devices are described.
\end{abstract}

\begin{keywords}
  MEMS microphone, SD card, PPG, EDR, Wi-Fi, Bluetooth, Lithium-Ion Battery, USB C Interface
\end{keywords}


%--- PRIVITCI / APPENDIX -------------------------------------------------------

% Sva poglavlja koja slijede će biti označena slovom i riječi privitak / All following chapters will be denoted with an appendix and a letter
\backmatter



\end{document}
